%%%%%%%%%%%%%%%%%%%%%%%%%%%%%%%%%%%%%%%%%
% Short Sectioned Assignment
% LaTeX Template
% Version 1.0 (5/5/12)
%
% This template has been downloaded from:
% http://www.LaTeXTemplates.com
%
% Original author:
% Frits Wenneker (http://www.howtotex.com)
%
% License:
% CC BY-NC-SA 3.0 (http://creativecommons.org/licenses/by-nc-sa/3.0/)
%
%%%%%%%%%%%%%%%%%%%%%%%%%%%%%%%%%%%%%%%%%

%----------------------------------------------------------------------------------------
%	PACKAGES AND OTHER DOCUMENT CONFIGURATIONS
%----------------------------------------------------------------------------------------

\documentclass[paper=a4, fontsize=11pt]{scrartcl} % A4 paper and 11pt font size

\usepackage[T1]{fontenc} % Use 8-bit encoding that has 256 glyphs
%\usepackage{fourier} % Use the Adobe Utopia font for the document - comment this line to return to the LaTeX default
\usepackage[english]{babel} % English language/hyphenation
\usepackage{amsmath,amsfonts,amsthm} % Math packages

%\usepackage{lipsum} % Used for inserting dummy 'Lorem ipsum' text into the template

\usepackage{usecases}

\usepackage{sectsty} % Allows customizing section commands
\sectionfont{\normalfont\scshape\textbf }
\subsectionfont{ \normalfont\scshape\textbf}

\usepackage{fancyhdr} % Custom headers and footers
\pagestyle{fancyplain} % Makes all pages in the document conform to the custom headers and footers
\fancyhead{} % No page header - if you want one, create it in the same way as the footers below
\fancyfoot[L]{} % Empty left footer
\fancyfoot[C]{} % Empty center footer
\fancyfoot[R]{\thepage} % Page numbering for right footer
\renewcommand{\headrulewidth}{0pt} % Remove header underlines
\renewcommand{\footrulewidth}{0pt} % Remove footer underlines
\setlength{\headheight}{13.6pt} % Customize the height of the header

\numberwithin{equation}{section} % Number equations within sections (i.e. 1.1, 1.2, 2.1, 2.2 instead of 1, 2, 3, 4)
\numberwithin{figure}{section} % Number figures within sections (i.e. 1.1, 1.2, 2.1, 2.2 instead of 1, 2, 3, 4)
\numberwithin{table}{section} % Number tables within sections (i.e. 1.1, 1.2, 2.1, 2.2 instead of 1, 2, 3, 4)

\setlength\parindent{0pt} % Removes all indentation from paragraphs - comment this line for an assignment with lots of text

%----------------------------------------------------------------------------------------
%	TITLE SECTION
%----------------------------------------------------------------------------------------

\newcommand{\horrule}[1]{\rule{\linewidth}{#1}} % Create horizontal rule command with 1 argument of height

\title{	
\normalfont \normalsize 
\textsc{International Institute of Information Technology, Bangalore} \\ [25pt] % Your university, school and/or department name(s)
\horrule{0.5pt} \\[0.4cm] % Thin top horizontal rule
\huge Smart Water Networks - Project Scope \\ % The assignment title
\horrule{2pt} \\[0.5cm] % Thick bottom horizontal rule
}

\author{Abhijith Madhav \and Kumudini Kakwani} % Your name

\date{\normalsize\today} % Today's date or a custom date

\begin{document}

\maketitle % Print the title

%----------------------------------------------------------------------------------------
%	Project Scope
%----------------------------------------------------------------------------------------

\section{Problem}
Water needed for the IIIT-B campus is sourced in three ways
\begin{enumerate}
\item
IIITB has its own source of water in the form of three functional borewells. Water is pumped out of these borewells for almost twenty hours a day. 
\item
Water supply from BWSSB for a limited amount of time each day.
\item
Almost 20000-30000 liters of water per day is procurred from commercial water tankers.
\end{enumerate}
Currently there is no insight into how water is being used and about whether its use is optimal or not. Informal estimates term the per-capita water consumption within the campus as excessive.

\section{Objective}
There is a proposal to make the water distribution network of the IIIT-B campus a smart one with the installation of sensors in the network. Our system intends to plug into this smart network and work on the data that the sensors produce. \\

Specifically this project intends to
\begin{enumerate}
\item
Be general purpose enough to plug into any small to medium sized smart water network(An institute, a corporate campus, a gated community, a farm etc)
\item
Help in better water management by offering control and insights into the water network across the installation.
\end{enumerate}

\section{Scope}

We will be focusing on architecting, designing and implementing the below
\begin{enumerate}
\item
Supervisor mobile apps which
\begin{itemize}
\item
Detail water usage patterns.
\item
Help in monitoring the status of resolution of anomalous events in the water network by the field personnal.
\end{itemize}
\item
Mobile apps for the field personnal to notify them of anomalous events in the smart water network which need attention.
\item
Architecting and implementing a modular backend storage system to
\begin{itemize}
\item
Represent the smart water network.
\item
Manage all the data generated by the sensors.\\
\end{itemize}
\end{enumerate}
We are limiting the extent of our implementation to \textbf{not} include the below
\begin{enumerate}
\item
Web UI's to construct the smart water network. We will load the structure of the smart water network in batch mode to our backend systems and then work on the same.
\item
A visual geospatial representation of the water network where a centralized view of the whole network is possible. Here each network asset has a visual representation and its parameters can be inspected by clicking on them.
\end{enumerate}

\section{Use cases}
\begin{usecase}
\addtitle{Use Case 1}{Insights into the water network}
\addfield{Actors}{Administrators}
\additemizedfield{Activities}{
\item Will be able to see the breakdown of water usage across aggregations with options to drill down to specific granularity. Aggregations can be buildings, blocks, floors etc. Aggregations can be nested.
\item Will be able to inspect health parameters of the water network. Health parameters can quality of water, electricity used, level of water in storage etc.
}
\end{usecase}

\begin{usecase}
\addtitle{Use Case 2}{Predictions}
\addfield{Actors}{Administrators}
\additemizedfield{Activities}{
\item Will get predictions regarding future water usage.
}
\end{usecase}

\begin{usecase}
\addtitle{Use Case 3}{Anomaly notifications}
\addfield{Actors}{Administrators, Field Staff}
\additemizedfield{Activities}{
\item Field staff will be notified of specific anomalies in the water network.
\item Administrators will be able to aggregate alerts w.r.t the type and origin and assign them to the field staff.
\item Administrators will be able to monitor resolution of anomalies by field staff.
}
\end{usecase}

\begin{usecase}
\addtitle{Use case 4}{Reporting leaks}
\addfield{Actors}{General populace}
\additemizedfield{Activities}{
\item Will be able to report leaks.
}
\end{usecase}

\section{Key features and functionalities}

\subsection{Android Application for the supervisor}

\begin{enumerate}
\item 
Gets Notifications
\begin{enumerate}
\item
When there are leaks(Leak detection)...
\item
When to water the garden...
\item
When quality of water goes down below a certain level...
\item
When level of water goes below a certain level in storage or sources...
\item
When water consumption increases beyond a certain in level...
\end{enumerate}
\item
Can subscribe field staff to relevent alerts and notifications.
\item
Can track notification/alert resolution by field staff.
\item
Reports
\begin{enumerate}
\item
Water consumption pattern with options to drill down w.r.t to buildings(Academic, Cafeteria, Hostels etc) and activities(Cooking, gardening, cleaning etc).
\item
Water consumption vs time vs number of students.
\end{enumerate}
\item
Predictions
\begin{enumerate}
\item
Water tanker requirement prediction.
\end{enumerate}
\end{enumerate}

\subsection{Android application for field staff}
\begin{enumerate}
\item
Customized notifications and options to update status of resolution.
\end{enumerate}

\subsection{Android application for the general populace}
\begin{enumerate}
\item
Report leaks
\end{enumerate}

\subsection{Basic Web Application for the supervisor}
The supervisor can then obtain relevent details for each network asset like the below with options to drill down w.r.t. time.
\begin{enumerate}
\item
Quality of water
\item
Storage levels
\item
Consumption of water
\item
Status information of pumps, i.e., whether they are switched on or switched off.
\item
Electricity consumed to pump water.
\end{enumerate}

\section{External Dependencies}
\begin{enumerate}
\item
Installation of flow, quality and level sensors at suitable points in the water network to make it a smart one. 
\item
Installation of a SCADA like system with a ODMS(A historian) from where sensor data can be read off.
\end{enumerate}
In the absence of the above we intend to implement a stop-gap sensors simulator which will fill database with psuedo sensor data until real sensors are deployed in the network.
\end{document}