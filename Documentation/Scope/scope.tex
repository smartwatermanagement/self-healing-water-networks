%%%%%%%%%%%%%%%%%%%%%%%%%%%%%%%%%%%%%%%%%
% Short Sectioned Assignment
% LaTeX Template
% Version 1.0 (5/5/12)
%
% This template has been downloaded from:
% http://www.LaTeXTemplates.com
%
% Original author:
% Frits Wenneker (http://www.howtotex.com)
%
% License:
% CC BY-NC-SA 3.0 (http://creativecommons.org/licenses/by-nc-sa/3.0/)
%
%%%%%%%%%%%%%%%%%%%%%%%%%%%%%%%%%%%%%%%%%

%----------------------------------------------------------------------------------------
%	PACKAGES AND OTHER DOCUMENT CONFIGURATIONS
%----------------------------------------------------------------------------------------

\documentclass[paper=a4, fontsize=11pt]{scrartcl} % A4 paper and 11pt font size

\usepackage[T1]{fontenc} % Use 8-bit encoding that has 256 glyphs
%\usepackage{fourier} % Use the Adobe Utopia font for the document - comment this line to return to the LaTeX default
\usepackage[english]{babel} % English language/hyphenation
\usepackage{amsmath,amsfonts,amsthm} % Math packages

%\usepackage{lipsum} % Used for inserting dummy 'Lorem ipsum' text into the template

\usepackage{sectsty} % Allows customizing section commands
\sectionfont{\normalfont\scshape\textbf }
\subsectionfont{ \normalfont\scshape}

\usepackage{fancyhdr} % Custom headers and footers
\pagestyle{fancyplain} % Makes all pages in the document conform to the custom headers and footers
\fancyhead{} % No page header - if you want one, create it in the same way as the footers below
\fancyfoot[L]{} % Empty left footer
\fancyfoot[C]{} % Empty center footer
\fancyfoot[R]{\thepage} % Page numbering for right footer
\renewcommand{\headrulewidth}{0pt} % Remove header underlines
\renewcommand{\footrulewidth}{0pt} % Remove footer underlines
\setlength{\headheight}{13.6pt} % Customize the height of the header

\numberwithin{equation}{section} % Number equations within sections (i.e. 1.1, 1.2, 2.1, 2.2 instead of 1, 2, 3, 4)
\numberwithin{figure}{section} % Number figures within sections (i.e. 1.1, 1.2, 2.1, 2.2 instead of 1, 2, 3, 4)
\numberwithin{table}{section} % Number tables within sections (i.e. 1.1, 1.2, 2.1, 2.2 instead of 1, 2, 3, 4)

\setlength\parindent{0pt} % Removes all indentation from paragraphs - comment this line for an assignment with lots of text

%----------------------------------------------------------------------------------------
%	TITLE SECTION
%----------------------------------------------------------------------------------------

\newcommand{\horrule}[1]{\rule{\linewidth}{#1}} % Create horizontal rule command with 1 argument of height

\title{	
\normalfont \normalsize 
\textsc{International Institute of Information Technology, Bangalore} \\ [25pt] % Your university, school and/or department name(s)
\horrule{0.5pt} \\[0.4cm] % Thin top horizontal rule
\huge Smart Water Networks - Project Scope \\ % The assignment title
\horrule{2pt} \\[0.5cm] % Thick bottom horizontal rule
}

\author{Abhijith Madhav \and Kumudini Kakwani} % Your name

\date{\normalsize\today} % Today's date or a custom date

\begin{document}

\maketitle % Print the title

%----------------------------------------------------------------------------------------
%	Project Scope
%----------------------------------------------------------------------------------------

\section{Sensor Data Simulator}
\begin{enumerate}
\item
Fill database with psuedo sensor data.
\end{enumerate}

\section{Web Application}

\subsection{Admin Console}
\begin{enumerate}
\item
Creating a representation of the water network for the system.
\item 
Getting Notifications
\begin{enumerate}
\item
When there are leaks...
\item
When to water the garden...
\item
When quality of water goes down below a certain level...
\item
When level of water goes below a certain level in storage or sources...
\item
When water consumption increases beyond a certain in level...
\end{enumerate}
\item
Subscribing field staff to relevent alerts and notifications.
\item
Tracking of notification/alert resolution by field staff.
\end{enumerate}

\subsection{ Dashboard }
\begin{enumerate}
\item
Reports
\begin{enumerate}
\item
Water consumption pattern with options to drill down w.r.t to buildings(Academic, Cafeteria, Hostels etc) and activities(Cooking, gardening, cleaning etc).
\item
Water consumption vs time vs number of students.
\end{enumerate}
\item
Predictions
\begin{enumerate}
\item
Water tanker requirement prediction.
\end{enumerate}
\end{enumerate}

\subsection{Geospatial representation of water network}
Graphical representation of the whole water network. Relevent details for each network asset like the below with options to drill down w.r.t. time.
\begin{enumerate}
\item
Quality of water
\item
Storage levels
\item
Consumption of water
\item
Status information of pumps, i.e., whether they are switched on or switched off.
\item
Electricity consumed to pump water.
\end{enumerate}



\section{Android Application}
\subsection{For general populace}
\begin{enumerate}
\item
Report leaks and individual water usage pattern.
\newline
This is planned for the case where there aren't extensive usage collection sensors throughout the network. This will be a mobile interface through which actors are going to be submitting data about their normal usage. The submission of data need not be done on a daily basis. The actors can submit data detailing their regular activities and the typical water consumption for each activity, say, 2 buckets for bath daily and 5 buckets for washing clothes once in three days. They will be able to link their usage behaviour to specific time intervals(typically days, weeks or months)
\end{enumerate}

\subsection{For field staff}
\begin{enumerate}
\item
Customized notifications and options to update status of resolution.
\end{enumerate}

\end{document}