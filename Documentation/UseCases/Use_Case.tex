%%%%%%%%%%%%%%%%%%%%%%%%%%%%%%%%%%%%%%%%%
% Short Sectioned Assignment
% LaTeX Template
% Version 1.0 (5/5/12)
%
% This template has been downloaded from:
% http://www.LaTeXTemplates.com
%
% Original author:
% Frits Wenneker (http://www.howtotex.com)
%
% License:
% CC BY-NC-SA 3.0 (http://creativecommons.org/licenses/by-nc-sa/3.0/)
%
%%%%%%%%%%%%%%%%%%%%%%%%%%%%%%%%%%%%%%%%%

%----------------------------------------------------------------------------------------
%	PACKAGES AND OTHER DOCUMENT CONFIGURATIONS
%----------------------------------------------------------------------------------------

\documentclass[paper=a4, fontsize=11pt]{scrartcl} % A4 paper and 11pt font size

\usepackage{usecases}
\usepackage[T1]{fontenc} % Use 8-bit encoding that has 256 glyphs
%\usepackage{fourier} % Use the Adobe Utopia font for the document - comment this line to return to the LaTeX default
\usepackage[english]{babel} % English language/hyphenation
\usepackage{amsmath,amsfonts,amsthm} % Math packages

\usepackage{lipsum} % Used for inserting dummy 'Lorem ipsum' text into the template

\usepackage{sectsty} % Allows customizing section commands
\allsectionsfont{\centering \normalfont\scshape} % Make all sections centered, the default font and small caps

\usepackage{fancyhdr} % Custom headers and footers
\pagestyle{fancyplain} % Makes all pages in the document conform to the custom headers and footers
\fancyhead{} % No page header - if you want one, create it in the same way as the footers below
\fancyfoot[L]{} % Empty left footer
\fancyfoot[C]{} % Empty center footer
\fancyfoot[R]{\thepage} % Page numbering for right footer
\renewcommand{\headrulewidth}{0pt} % Remove header underlines
\renewcommand{\footrulewidth}{0pt} % Remove footer underlines
\setlength{\headheight}{13.6pt} % Customize the height of the header

\numberwithin{equation}{section} % Number equations within sections (i.e. 1.1, 1.2, 2.1, 2.2 instead of 1, 2, 3, 4)
\numberwithin{figure}{section} % Number figures within sections (i.e. 1.1, 1.2, 2.1, 2.2 instead of 1, 2, 3, 4)
\numberwithin{table}{section} % Number tables within sections (i.e. 1.1, 1.2, 2.1, 2.2 instead of 1, 2, 3, 4)

\setlength\parindent{0pt} % Removes all indentation from paragraphs - comment this line for an assignment with lots of text

%----------------------------------------------------------------------------------------
%	TITLE SECTION
%----------------------------------------------------------------------------------------

\newcommand{\horrule}[1]{\rule{\linewidth}{#1}} % Create horizontal rule command with 1 argument of height

\title{	
\normalfont \normalsize 
\textsc{International Institute of Information Technology, Bangalore} \\ [25pt] % Your university, school and/or department name(s)
\horrule{0.5pt} \\[0.4cm] % Thin top horizontal rule
\huge Self Healing Water Networks \\ % The assignment title
\horrule{2pt} \\[0.5cm] % Thick bottom horizontal rule
}

\author{Kumudini Kakwani 
\and Arjun S Bharadwaj
\and Abhijith Madhav}

\date{\normalsize\today} % Today's date or a custom date

\begin{document}

\maketitle % Print the title

\section{Assumptions}
\begin{itemize}
\item Water level sensors in the borewell, quality sensors in the borewell, flow/pressure sensors at various points in the networks
\item Scada like system with a ODMS(A historian)
\end{itemize}

\section{Use Cases}

\begin{usecase}
\addtitle{Use Case 1}{Dashboard of water usage and related patterns}
\addfield{Actors}{Management, Admins}
\additemizedfield{Activities}{
\item Will be presented with a dashboard on water usage patterns across buildings(hostels, Academic block, cafeteria) and activities(Cooking, gardening, cleaning, etc) with options to drill down to specific granularity.
\item Reports on electricity consumption due to pumping.
\item Reports on water consumption vs number of students in campus.
\item Analysis of peaks and troughs in the usage of water.
\item Water usage in the campus w.r.t weather
\item Usage pattern of water throughout the day and hence pressure at which water needs to be pumped.
\item Water requirement prediction in the coming days vs predicted levels in the storage.
}
\end{usecase}

\begin{usecase}
\addtitle{Use case 2}{Customizable Alerts and Notifications}
\addfield{Actors}{ Admin/Estate manager}
\additemizedfield{Activities}{
\item Will be provided with alerts for
\begin{itemize}
\item Probable leaks
\item Inefficient pumping
\item Water contamination or excessive chemical levels in water
\item Inefficient pressure or excessive pressure at outlet points
\end{itemize}

}
\end{usecase}

\begin{usecase}
\addtitle{Use case 3}{ Input usage data through mobile/web interface}
\addfield{Actors}{ Student, Housekeeping incharge}
\additemizedfield{Activities}{
\item This is planned for the case where there aren't extensive usage collection sensors throughout the network.
\item This will be a web/mobile interface through which actors are going to be submitting data about their normal usage.
\item The submission of data need not be done on a daily basis. The actors can submit data detailing their regular activities and the typical water consumption for each, say 2 buckets for bath daily, 5 buckets for washing clothes once in three days. They will be able to link their usage behaviour to specific time intervals(typically days, weeks or months)
}
\end{usecase}

\begin{usecase}
\addtitle{Use case 4}{ Reporting leaks through mobile/web interface}
\addfield{Actors}{Student}
\additemizedfield{Activities}{
\item Students can report leaks in the respective washrooms.
}
\end{usecase}

\begin{usecase}
\addtitle{Use case 5}{A geospatial representation of the water distribution system}
\addfield{Actors}{Admin/Estate Manager}
\additemizedfield{Activities}{
\item Graphical representation of whole water network.
\item Key health or performance indicators to represent status of each asset. Example pressure or flow rate of water flowing through a section of the pipe. 
\item Ability to access historical data for each asset.
\item Role based access: Different users have different privileges and access to data based on their role.
}
\end{usecase}

\begin{usecase}
\addtitle{Use case 6}{Notifications for watering of plants}
\addfield{Actors}{Housekeeping Staff}
\additemizedfield{Activities}{
\item Based on data from soil sensor and weather forecast, send notifications for watering the plants.
}
\end{usecase}

\begin{usecase}
\addtitle{Use case 7}{Issue Tracker}
\addfield{Actors}{Students, Admin, Staff}
\additemizedfield{Activities}{
\item Reporting and tracking of issues.
}
\end{usecase}

\end{document}