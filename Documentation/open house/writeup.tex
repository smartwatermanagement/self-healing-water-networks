%%%%%%%%%%%%%%%%%%%%%%%%%%%%%%%%%%%%%%%%%
% Short Sectioned Assignment
% LaTeX Template
% Version 1.0 (5/5/12)
%
% This template has been downloaded from:
% http://www.LaTeXTemplates.com
%
% Original author:
% Frits Wenneker (http://www.howtotex.com)
%
% License:
% CC BY-NC-SA 3.0 (http://creativecommons.org/licenses/by-nc-sa/3.0/)
%
%%%%%%%%%%%%%%%%%%%%%%%%%%%%%%%%%%%%%%%%%

%----------------------------------------------------------------------------------------
%	PACKAGES AND OTHER DOCUMENT CONFIGURATIONS
%----------------------------------------------------------------------------------------

\documentclass[paper=a4, fontsize=11pt]{scrartcl} % A4 paper and 11pt font size

\usepackage[T1]{fontenc} % Use 8-bit encoding that has 256 glyphs
%\usepackage{fourier} % Use the Adobe Utopia font for the document - comment this line to return to the LaTeX default
\usepackage[english]{babel} % English language/hyphenation
\usepackage{amsmath,amsfonts,amsthm} % Math packages

%\usepackage{lipsum} % Used for inserting dummy 'Lorem ipsum' text into the template

%\usepackage{usecases}

%\usepackage{sectsty} % Allows customizing section commands
%\sectionfont{\normalfont\scshape\textbf }
%\subsectionfont{ \normalfont\scshape\textbf}

\usepackage{fancyhdr} % Custom headers and footers
\pagestyle{fancyplain} % Makes all pages in the document conform to the custom headers and footers
\fancyhead{} % No page header - if you want one, create it in the same way as the footers below
\fancyfoot[L]{} % Empty left footer
\fancyfoot[C]{} % Empty center footer
\fancyfoot[R]{\thepage} % Page numbering for right footer
\renewcommand{\headrulewidth}{0pt} % Remove header underlines
\renewcommand{\footrulewidth}{0pt} % Remove footer underlines
\setlength{\headheight}{13.6pt} % Customize the height of the header

\numberwithin{equation}{section} % Number equations within sections (i.e. 1.1, 1.2, 2.1, 2.2 instead of 1, 2, 3, 4)
\numberwithin{figure}{section} % Number figures within sections (i.e. 1.1, 1.2, 2.1, 2.2 instead of 1, 2, 3, 4)
\numberwithin{table}{section} % Number tables within sections (i.e. 1.1, 1.2, 2.1, 2.2 instead of 1, 2, 3, 4)

\setlength\parindent{0pt} % Removes all indentation from paragraphs - comment this line for an assignment with lots of text

%----------------------------------------------------------------------------------------
%	TITLE SECTION
%----------------------------------------------------------------------------------------

\newcommand{\horrule}[1]{\rule{\linewidth}{#1}} % Create horizontal rule command with 1 argument of height

\title{	
\normalfont \normalsize 
\textsc{International Institute of Information Technology, Bangalore} \\ [25pt] % Your university, school and/or department name(s)
\horrule{0.5pt} \\[0.4cm] % Thin top horizontal rule
\huge Smart Water Networks \\ % The assignment title
\horrule{2pt} \\[0.5cm] % Thick bottom horizontal rule
}

\author{Abhijith Madhav \and Kumudini Kakwani} % Your name

\date{\normalsize\today} % Today's date or a custom date

\begin{document}

\maketitle % Print the title

%----------------------------------------------------------------------------------------
%	Project Scope
%----------------------------------------------------------------------------------------


Water needed for the IIIT-B campus is sourced in three ways
\begin{enumerate}
\item
IIITB has its own source of water in the form of three functional borewells. Water is pumped out of these borewells for almost twenty hours a day. 
\item
Water supply from BWSSB for a limited amount of time each day.
\item
Almost 20000-30000 liters of water per day is procurred from commercial water tankers.
\end{enumerate}
Currently there is no insight into how water is being used and about whether its use is optimal or not. Informal estimates term the per-capita water consumption within the campus as excessive.\\

There is a proposal to make the water distribution network of the IIIT-B campus a smart one with the installation of sensors in the network. Our system intends to plug into this smart network and work on the data that the sensors produce. \\

The smart water network consists of network assets like storages, pipes and outlets. Each of these assets may have sensors deployed in them which provide with useful metrics like flow data, data about the quantity and quality of water etc. Our system has a representation of this smart water network as a part of its application backend. \\

We also have a monitoring system which continuously monitors the data produced by sensors all over the network. It then produces notifications when it detects anomalous events. Apart from these anomalous events, the monitoring system also runs algorithms to predict water requirement and to detect leaks.

In all our system provides API's to accomplish the following.
\begin{itemize}
\item Get notifications about issues detected by the monitoring system.
\item Discover the structure of the water network.
\item Get water usage breakup across the network and water usage trends over time.
\end{itemize}

We have build an android application which consumes the above API's to provide the following features
\begin{itemize}
\item Notifications about threshold breaches, leaks, water requirement prediction for next day and watering garden suggestions.
\item Reports on water usage across network aggregations in the specified time frame. Roll up and drill down possible.
\item Hierarchical view of the network to get an overview of the health in terms of issues .
\end{itemize}
\end{document}